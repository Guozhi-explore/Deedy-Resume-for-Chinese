%%%%%%%%%%%%%%%%%%%%%%%%%%%%%%%%%%%%%%%
% Deedy - One Page Two Column Resume
% LaTeX Template
% Version 1.2 (16/9/2014)
%
% Original author:
% Debarghya Das (http://debarghyadas.com)
%
% Original repository:
% https://github.com/deedydas/Deedy-Resume
%
% IMPORTANT: THIS TEMPLATE NEEDS TO BE COMPILED WITH XeLaTeX
%
% This template uses several fonts not included with Windows/Linux by
% default. If you get compilation errors saying a font is missing, find the line
% on which the font is used and either change it to a font included with your
% operating system or comment the line out to use the default font.
% 
%%%%%%%%%%%%%%%%%%%%%%%%%%%%%%%%%%%%%%
% 
% TODO:
% 1. Integrate biber/bibtex for article citation under publications.
% 2. Figure out a smoother way for the document to flow onto the next page.
% 3. Add styling information for a "Projects/Hacks" section.
% 4. Add location/address information
% 5. Merge OpenFont and MacFonts as a single sty with options.
% 
%%%%%%%%%%%%%%%%%%%%%%%%%%%%%%%%%%%%%%
%
% CHANGELOG:
% v1.1:
% 1. Fixed several compilation bugs with \renewcommand
% 2. Got Open-source fonts (Windows/Linux support)
% 3. Added Last Updated
% 4. Move Title styling into .sty
% 5. Commented .sty file.
%
%%%%%%%%%%%%%%%%%%%%%%%%%%%%%%%%%%%%%%%
%
% Known Issues:
% 1. Overflows onto second page if any column's contents are more than the
% vertical limit
% 2. Hacky space on the first bullet point on the second column.
%
%%%%%%%%%%%%%%%%%%%%%%%%%%%%%%%%%%%%%%


\documentclass[]{deedy-resume-openfont}
\usepackage{fancyhdr}
    
\pagestyle{fancy}
\fancyhf{}
    
\begin{document}

%%%%%%%%%%%%%%%%%%%%%%%%%%%%%%%%%%%%%%
%
%     LAST UPDATED DATE
%
%%%%%%%%%%%%%%%%%%%%%%%%%%%%%%%%%%%%%%
\lastupdated

%%%%%%%%%%%%%%%%%%%%%%%%%%%%%%%%%%%%%%
%
%     TITLE NAME
%
%%%%%%%%%%%%%%%%%%%%%%%%%%%%%%%%%%%%%%
\namesection{郭}{志}{ \urlstyle{same}\href{mailto:qtxuning1999@sjtu.edu.cn}{qtxuning1999@sjtu.edu.cn}
}

%%%%%%%%%%%%%%%%%%%%%%%%%%%%%%%%%%%%%%
%
%     COLUMN ONE
%
%%%%%%%%%%%%%%%%%%%%%%%%%%%%%%%%%%%%%%

\begin{minipage}[t]{0.25\textwidth} 

\section{个人信息}
sectionsep
\descript{姓名: 郭志}
\descript{出生年月: 1999/12}
\descript{居住地: 上海闵行}
\descript{Tel: 18621813268}
\descript{Mail: qtxuning1999@sjtu.edu.cn}
\sectionsep
%%%%%%%%%%%%%%%%%%%%%%%%%%%%%%%%%%%%%%
%     EDUCATION
%%%%%%%%%%%%%%%%%%%%%%%%%%%%%%%%%%%%%%

\section{教育经历} 
\sectionsep

\subsection{上海交通大学}
\descript{软件学院软件工程}
\location{2019.02-2021.07}
\sectionsep

\subsection{上海交通大学}
\descript{电子信息与电气工程学院}
\descript{自动化系}
\location{2017.07-2019.02}
\sectionsep

%%%%%%%%%%%%%%%%%%%%%%%%%%%%%%%%%%%%%%
%     LINKS
%%%%%%%%%%%%%%%%%%%%%%%%%%%%%%%%%%%%%%

\section{链接}
\sectionsep  
Github:// \href{https://github.com/Guozhi-explore}{\bf Guozhi} \\
LinkedIn://  \href{https://www.linkedin.com/in/zhi-guo-316892186}{\bf Guozhi} \\
\sectionsep

%%%%%%%%%%%%%%%%%%%%%%%%%%%%%%%%%%%%%%
%     COURSEWORK
%%%%%%%%%%%%%%%%%%%%%%%%%%%%%%%%%%%%%%

\section{修读课程}
\sectionsep
计算机系统基础 93,97,92 \\
软件基础实践 99\\
计算机系统工程 85 \\
编译原理与技术 95 \\
Web开发 93 \\
数据结构 91 \\
\end{tightemize}
 \sectionsep
 \sectionsep
%%%%%%%%%%%%%%%%%%%%%%%%%%%%%%%%%%%%%%
%     SKILLS
%%%%%%%%%%%%%%%%%%%%%%%%%%%%%%%%%%%%%%

\section{技能}
\sectionsep
\subsection{编程语言}
\location{超过 10000 行}
C++ \\
\location{1000 - 5000 行}
C \textbullet{} Java \\
\location{低于 1000 行}
Python \textbullet{} Javascript \textbullet{} \LaTeX\ \textbullet{} Verilog \textbullet{} HTML \\ 
\sectionsep

\subsection{开发}
\location{一般}
SpringBoot  \textbullet{} MySQL \\
\location{了解}
Nginx \textbullet{} Docker \textbullet{} Mahout \textbullet{} MongoDB \textbullet{} Vue \textbullet{} Android \textbullet{} Zookeeper \textbullet{} Socket\\


%%%%%%%%%%%%%%%%%%%%%%%%%%%%%%%%%%%%%%
%
%     COLUMN TWO
%
%%%%%%%%%%%%%%%%%%%%%%%%%%%%%%%%%%%%%%

\end{minipage} 
\hfill
\begin{minipage}[t]{0.73\textwidth} 

%%%%%%%%%%%%%%%%%%%%%%%%%%%%%%%%%%%%%%
%     EXPERIENCE
%%%%%%%%%%%%%%%%%%%%%%%%%%%%%%%%%%%%%%

\section{项目}
\sectionsep
\runsubsection{\href{https://github.com/Guozhi-explore/Compilers}Tiger语言编译器}
\descript{课程个人作业}
\location{2019.09 - 2019.12}
\vspace{\topsep}
\begin{tightemize}
    \item 实现了从词法分析到x86-64代码生成的完整编译器\textbf{答辩得分 39/40}
    \item 使用 flexc++完成词法分析,bisonc++进行语法分析
    \item 用启发式的着色算法进行寄存器分配
    \item 生成了符合x86-64标准的汇编代码
\end{tightemize}
\sectionsep
\sectionsep

\runsubsection{网络文件系统}
\descript{课程个人作业}
\location{2019.09-2019.11}
\begin{tightemize}
\item 实现了文件系统的基本接口
\item 实现了客户端通过RPC读写文件的File Server
\item 通过Lock Server控制锁实现并发
\item 保证一致性的前提下,客户端可以对文件和锁在本地进行缓存
\end{tightemize}
\sectionsep
\sectionsep

\runsubsection{交大课程交流平台}
\descript{暑期团队作业 | 后端}
\location{2019.07-2019.08}
\begin{tightemize}
\item 使用Spring-Boot框架进行分层架构
\item Spring-Data JPA 操作 MySQL与MongoDB
\item JUnit 单元测试
\item 协同过滤算法进行课程推荐
\end{tightemize}
\sectionsep
\sectionsep

\runsubsection{深入理解计算机系统课程实验}
\descript{课程实验 | 10个}
\location{2019.03-2019.12}
\begin{tightemize}
\item 实现了web proxy,shell,memory allocator,cpu pipeline等项目
\item 掌握了并发编程,网络,异常,汇编等重要的系统编程技能
\end{tightemize}
\sectionsep
\sectionsep

\runsubsection{AI贪吃蛇}
\descript{课程个人作业 | 满分}
\location{2019.06-2019.06}
\begin{tightemize}
\item 使用Qt Gui
\item 实现了双人对战,人机PK等功能
\item 采用BFS算法进行寻路
\item 面向对象开发
\end{tightemize}
\sectionsep
\sectionsep

\runsubsection{扫地机器人手机遥控程序}
\descript{比赛项目}
\location{2018.02-2018.06}
\begin{tightemize}
\item android app开发
\item 通过蓝牙与扫地机器人通信控制移动
\end{tightemize}
\sectionsep
\sectionsep



%%%%%%%%%%%%%%%%%%%%%%%%%%%%%%%%%%%%%%
%     AWARDS
%%%%%%%%%%%%%%%%%%%%%%%%%%%%%%%%%%%%%%

\section{所获奖项} 
\begin{tabular}{rll}
2018         & 奖学金  & 国家励志奖学金 \\
2018	     & 奖学金  & 上海交通大学B类奖学金 \\
2018	     & 二等奖  & 上海交通大学扫地机器人大赛 \\
2019         & 三等奖  & 上海交通大学软件展示会\\
\end{tabular}
\sectionsep

%%%%%%%%%%%%%%%%%%%%%%%%%%%%%%%%%%%%%%
%     PUBLICATIONS
%%%%%%%%%%%%%%%%%%%%%%%%%%%%%%%%%%%%%%

% \section{Publications} 
% \renewcommand\refname{\vskip -1.5cm} % Couldn't get this working from the .cls file
% \bibliographystyle{abbrv}
% \bibliography{publications}
% \nocite{*}

\end{minipage} 
\end{document}  \documentclass[]{article}
