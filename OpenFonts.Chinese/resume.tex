%%%%%%%%%%%%%%%%%%%%%%%%%%%%%%%%%%%%%%%
% Deedy - One Page Two Column Resume
% LaTeX Template
% Version 1.2 (16/9/2014)
%
% Original author:
% Debarghya Das (http://debarghyadas.com)
%
% Original repository:
% https://github.com/deedydas/Deedy-Resume
%
% IMPORTANT: THIS TEMPLATE NEEDS TO BE COMPILED WITH XeLaTeX
%
% This template uses several fonts not included with Windows/Linux by
% default. If you get compilation errors saying a font is missing, find the line
% on which the font is used and either change it to a font included with your
% operating system or comment the line out to use the default font.
% 
%%%%%%%%%%%%%%%%%%%%%%%%%%%%%%%%%%%%%%
% 
% TODO:
% 1. Integrate biber/bibtex for article citation under publications.
% 2. Figure out a smoother way for the document to flow onto the next page.
% 3. Add styling information for a "Projects/Hacks" section.
% 4. Add location/address information
% 5. Merge OpenFont and MacFonts as a single sty with options.
% 
%%%%%%%%%%%%%%%%%%%%%%%%%%%%%%%%%%%%%%
%
% CHANGELOG:
% v1.1:
% 1. Fixed several compilation bugs with \renewcommand
% 2. Got Open-source fonts (Windows/Linux support)
% 3. Added Last Updated
% 4. Move Title styling into .sty
% 5. Commented .sty file.
%
%%%%%%%%%%%%%%%%%%%%%%%%%%%%%%%%%%%%%%%
%
% Known Issues:
% 1. Overflows onto second page if any column's contents are more than the
% vertical limit
% 2. Hacky space on the first bullet point on the second column.
%
%%%%%%%%%%%%%%%%%%%%%%%%%%%%%%%%%%%%%%


\documentclass[]{deedy-resume-openfont}
\usepackage{fancyhdr}
    
\pagestyle{fancy}
\fancyhf{}
    
\begin{document}

%%%%%%%%%%%%%%%%%%%%%%%%%%%%%%%%%%%%%%
%
%     LAST UPDATED DATE
%
%%%%%%%%%%%%%%%%%%%%%%%%%%%%%%%%%%%%%%
\lastupdated

%%%%%%%%%%%%%%%%%%%%%%%%%%%%%%%%%%%%%%
%
%     TITLE NAME
%
%%%%%%%%%%%%%%%%%%%%%%%%%%%%%%%%%%%%%%
\namesection{志}{郭}{ \urlstyle{same}\href{mailto:qtxuning1999@sjtu.edu.cn}{qtxuning1999@sjtu.edu.cn}
}

%%%%%%%%%%%%%%%%%%%%%%%%%%%%%%%%%%%%%%
%
%     COLUMN ONE
%
%%%%%%%%%%%%%%%%%%%%%%%%%%%%%%%%%%%%%%

\begin{minipage}[t]{0.25\textwidth} 

%%%%%%%%%%%%%%%%%%%%%%%%%%%%%%%%%%%%%%
%     EDUCATION
%%%%%%%%%%%%%%%%%%%%%%%%%%%%%%%%%%%%%%

\section{教育经历} 
\sectionsep

\subsection{上海交通大学}
\descript{软件学院软件工程}
\location{2019.02-2021.07}
\sectionsep

\subsection{上海交通大学}
\descript{电子信息与电气工程学院自动化系}
\location{2017.07-2019.02}
\sectionsep

%%%%%%%%%%%%%%%%%%%%%%%%%%%%%%%%%%%%%%
%     LINKS
%%%%%%%%%%%%%%%%%%%%%%%%%%%%%%%%%%%%%%

\section{链接}
\sectionsep  
Github:// \href{https://github.com/Guozhi-explore}{\bf Guozhi} \\
(380+ 关注者) \\
LinkedIn://  \href{https://www.linkedin.com/in/zhi-guo-316892186}{\bf Guozhi} \\

%%%%%%%%%%%%%%%%%%%%%%%%%%%%%%%%%%%%%%
%     COURSEWORK
%%%%%%%%%%%%%%%%%%%%%%%%%%%%%%%%%%%%%%

\section{修读课程}
\subsection{Graduate}
 计算机系统基础  93,97,92 \\
 软件基础实践 99\\
 计算机系统工程 85 \\
 编译原理与技术 95 \\
 Web开发 93 \\
 数据结构 91 \\
 \sectionsep

%%%%%%%%%%%%%%%%%%%%%%%%%%%%%%%%%%%%%%
%     SKILLS
%%%%%%%%%%%%%%%%%%%%%%%%%%%%%%%%%%%%%%

\section{技能}
\sectionsep
\subsection{编程}
\location{超过 5000 行}
Go \textbullet{} Python \textbullet{} C++ \textbullet{} Java \\
\location{1000 - 5000 行}
R \textbullet{} C \textbullet{} Scala \textbullet{} \LaTeX\ \\
\location{低于 1000 行}
HTML \textbullet{} Javascript \textbullet{} MatLab \textbullet{} Shell \textbullet{} Processing \\ 
\sectionsep

\subsection{云计算}
\location{一般}
Docker \textbullet{} Kubernetes \\
\location{了解}
Swarm \textbullet{} Moby \textbullet{} Linuxkit \textbullet{} HyperContainers \textbullet{} ClearContainers \textbullet{} Xen \textbullet{} KVM \textbullet{} Unikernel \\
\sectionsep

\subsection{DevOps}
\location{一般}
微服务 \textbullet{} Jenkins \textbullet{} Travis CI

%%%%%%%%%%%%%%%%%%%%%%%%%%%%%%%%%%%%%%
%
%     COLUMN TWO
%
%%%%%%%%%%%%%%%%%%%%%%%%%%%%%%%%%%%%%%

\end{minipage} 
\hfill
\begin{minipage}[t]{0.73\textwidth} 

%%%%%%%%%%%%%%%%%%%%%%%%%%%%%%%%%%%%%%
%     EXPERIENCE
%%%%%%%%%%%%%%%%%%%%%%%%%%%%%%%%%%%%%%

\section{实习经历}
\sectionsep
\runsubsection{谷歌编程之夏}
\descript{学生参与者}
\location{2017.05 - 2017.09 | 远程}
\vspace{\topsep}
\begin{tightemize}
    \item 共有 20651 个注册学生,其中 1318 个申请被谷歌接收,\textbf{接收率 6\%}
    \item 为 Processing 基金会实现了对 Processing 的 R 语言支持
    \item 与社区紧密合作,实现对 Processing 库的支持和对 R 包的支持
    \item 所做项目 \href{https://github.com/gaocegege/Processing.R}{\bf Processing.R} 在 GitHub 上获得 \textbf{70 stars},成为本次编程之夏 star 最多的项目
\end{tightemize}
\sectionsep

\runsubsection{摩根士丹利}
\descript{CIP 项目实习生}
\location{2017.02-2017.08 | 上海}
\begin{tightemize}
\item 优化开源容器调度管理框架 treadmill 的调度器
\item 实现与 Kubernetes 类似的调度模型,同时保留自身的树形结构
\end{tightemize}
\sectionsep

\runsubsection{上海触宝信息技术有限公司}
\descript{数据工程师(实习)}
\location{2015.09-2015.09 | 上海}
\begin{tightemize}
\item 移植爬虫代码到新的平台,优化重写部分过期的爬虫
\end{tightemize}
\sectionsep

\runsubsection{蚂蚁金服(杭州)网络技术有限公司}
\descript{Java 研发工程师(实习)}
\location{2015.07-2015.09 | 杭州}
\begin{tightemize}
\item 在支付宝国际事业团队从事海外直购业务开发
\item 实现部分包裹清关的逻辑和后台管理的逻辑
\end{tightemize}
\sectionsep

%%%%%%%%%%%%%%%%%%%%%%%%%%%%%%%%%%%%%%
%     RESEARCH
%%%%%%%%%%%%%%%%%%%%%%%%%%%%%%%%%%%%%%

\section{项目与论文}
\sectionsep
\runsubsection{\href{https://github.com/caicloud/cyclone}{\bf Cyclone}}
\descript{Maintainer}
\location{2016.11}
\begin{tightemize}
    \item 基于 Docker 的持续集成与持续部署系统
    \item 本科毕业设计,与才云科技合作开发,在 GitHub 上获得 \textbf{440 stars}
    \item 调研其他开源实现,确定工作流程和架构选型,实现 YAML parser 和 Docker 的运行时集成
    \end{tightemize}
\sectionsep

\runsubsection{\href{https://github.com/gaocegege/scrala}{\bf Scrala}}
\descript{Owner}
\location{2015.12}
\begin{tightemize}
    \item 使用 scala 实现的爬虫框架,灵感来自 scrapy
    \item 在 GitHub 上获得 \textbf{70 stars}
    \item 底层使用 Actor 模型取代 Python 中的异步模型
    \end{tightemize}
\sectionsep

%%%%%%%%%%%%%%%%%%%%%%%%%%%%%%%%%%%%%%
%     OPEN SOURCE
%%%%%%%%%%%%%%%%%%%%%%%%%%%%%%%%%%%%%%

\section{开源贡献}
\begin{tabular}{ll}
\href{https://github.com/moby/moby/commits?author=gaocegege}{\bf moby/moby} & 实现 docker service ps -q 参数,与 swarmkit 更好集成 \\
\href{https://github.com/opencontainers/runc/commits?author=gaocegege}{\bf opencontainers/runc} & 为了修复 \href{https://github.com/moby/moby/issues/27484}{moby/moby\#27484} 对上游进行的修改 \\
\href{https://github.com/pingcap/tidb/commits?author=gaocegege}{\bf pingcap/tidb} & 在 travis 里引入了覆盖率测试; 实现 truncate 函数 \\
\href{https://github.com/coala/coala-vs-code/commits/master?author=gaocegege}{\bf coala/coala-vs-code} & Visual Studio Code 上的插件,项目 maintainer \\
\href{https://github.com/weijianwen/SJTUThesis/commits?author=gaocegege}{\bf weijianwen/SJTUThesis} & 为学士论文模板添加英文大摘要; 替换版权字体 \\
\end{tabular}
\sectionsep

%%%%%%%%%%%%%%%%%%%%%%%%%%%%%%%%%%%%%%
%     AWARDS
%%%%%%%%%%%%%%%%%%%%%%%%%%%%%%%%%%%%%%

\section{所获奖项} 
\begin{tabular}{rll}
2017         & 奖学金  & 因特尔奖学金 \\
2016	     & 二等奖  & 第十三届全国研究生数学建模竞赛 \\
2016	     & 一等奖  & 第七届中国大学生服务外包创新创业大赛 \\
2015	     & 二等奖  & 美国大学生数学建模竞赛 \\
2015         & 一等奖 & 中国大学生数学建模竞赛上海赛区 \\
2014	     & 二等奖/杰出个人  & 大众点评校园 Hackathon \\
\end{tabular}
\sectionsep

%%%%%%%%%%%%%%%%%%%%%%%%%%%%%%%%%%%%%%
%     PUBLICATIONS
%%%%%%%%%%%%%%%%%%%%%%%%%%%%%%%%%%%%%%

% \section{Publications} 
% \renewcommand\refname{\vskip -1.5cm} % Couldn't get this working from the .cls file
% \bibliographystyle{abbrv}
% \bibliography{publications}
% \nocite{*}

\end{minipage} 
\end{document}  \documentclass[]{article}
